\documentclass[a4paper]{article}

%% Language and font encodings
\usepackage[italian]{babel}
\usepackage[utf8x]{inputenc}
\usepackage[T1]{fontenc}

%% Sets page size and margins
\usepackage[a4paper,top=3cm,bottom=2cm,left=3cm,right=3cm,marginparwidth=1.75cm]{geometry}

%% Useful packages
\usepackage{graphicx}
\usepackage{float}
\usepackage[colorlinks=true, allcolors=blue]{hyperref}

\title{Power}
\author{Yvette Chance}

\begin{document}
\maketitle

\begin{figure}[H]
	\centering
	\includegraphics[width=0.3\linewidth]{Power.jpg}
	\caption{Logo Power}
\end{figure}

\section{Descrizione}

Power permette di elevare a potenza qualsiasi coppia di numeri.
L'implementazione di Power è stata effettuata in linguaggio Java ed embeddata in Jolie.

\section{Operazioni}

Power espone una sola operazione di segnatura power( powr )( double ).

\section{Tipi}

Il tipo powr è la variante utilizzata da Power del tipo di autenticazione di APIMarket.
\begin{itemize}
	\item powr: void;
	\begin{itemize}
		\item .x: double;
		\item .y: double;
		\item .key: string;
		\item .user: string;
		\item .api: int.
	\end{itemize}
\end{itemize}

\end{document}